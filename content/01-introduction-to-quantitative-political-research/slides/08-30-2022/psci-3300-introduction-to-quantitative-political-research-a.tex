\documentclass[12pt,ignorenonframetext,aspectratio=169]{beamer}

\usepackage{pgfpages}
\setbeamertemplate{caption}[numbered]
\setbeamertemplate{caption label separator}{: }
\setbeamercolor{caption name}{fg=normal text.fg}
\beamertemplatenavigationsymbolsempty
% Prevent slide breaks in the middle of a paragraph
\widowpenalties 1 10000
\raggedbottom
\setbeamertemplate{part page}{
	\centering
	\begin{beamercolorbox}[sep=16pt,center]{part title}
		\usebeamerfont{part title}\insertpart\par
	\end{beamercolorbox}
}
\setbeamertemplate{section page}{
	\centering
	\begin{beamercolorbox}[sep=12pt,center]{part title}
		\usebeamerfont{section title}\insertsection\par
	\end{beamercolorbox}
}
\setbeamertemplate{subsection page}{
	\centering
	\begin{beamercolorbox}[sep=8pt,center]{part title}
		\usebeamerfont{subsection title}\insertsubsection\par
	\end{beamercolorbox}
}
\AtBeginPart{
	\frame{\partpage}
}
\AtBeginSection{
	\ifbibliography
	\else
	\frame{\sectionpage}
	\fi
}
\AtBeginSubsection{
	\frame{\subsectionpage}
}
\usepackage{amsmath,amssymb}
\usepackage{lmodern}
\usepackage{iftex}
\ifPDFTeX
\usepackage[T1]{fontenc}
\usepackage[utf8]{inputenc}
\usepackage{textcomp} % provide euro and other symbols
\else % if luatex or xetex
\usepackage{unicode-math}
\defaultfontfeatures{Scale=MatchLowercase}
\defaultfontfeatures[\rmfamily]{Ligatures=TeX,Scale=1}
\fi
% Use upquote if available, for straight quotes in verbatim environments
\IfFileExists{upquote.sty}{\usepackage{upquote}}{}
\IfFileExists{microtype.sty}{% use microtype if available
	\usepackage[]{microtype}
	\UseMicrotypeSet[protrusion]{basicmath} % disable protrusion for tt fonts
}{}
\makeatletter
\@ifundefined{KOMAClassName}{% if non-KOMA class
	\IfFileExists{parskip.sty}{%
		\usepackage{parskip}
	}{% else
		\setlength{\parindent}{0pt}
		\setlength{\parskip}{6pt plus 2pt minus 1pt}}
}{% if KOMA class
	\KOMAoptions{parskip=half}}
\makeatother
\usepackage{xcolor}
\IfFileExists{xurl.sty}{\usepackage{xurl}}{} % add URL line breaks if available
\IfFileExists{bookmark.sty}{\usepackage{bookmark}}{\usepackage{hyperref}}

\usepackage{csquotes}
\usepackage[style=chicago-authordate,backend=biber,sorting=nyt]{biblatex}

	\addbibresource{../../../../assets/references.bib}

%% Comment these out if you don't want a slide with just the
%% part/section/subsection/subsubsection title:
%\AtBeginPart{
%	\let\insertpartnumber\relax
%	\let\partname\relax
%	\frame{\partpage}
%}
%\AtBeginSection{
%	\let\insertsectionnumber\relax
%	\let\sectionname\relax
%	\frame{\sectionpage}
%}
%\AtBeginSubsection{
%	\let\insertsubsectionnumber\relax
%	\let\subsectionname\relax
%	\frame{\subsectionpage}
%}

\setlength{\emergencystretch}{3em}  % prevent overfull lines
\providecommand{\tightlist}{%
	\setlength{\itemsep}{0.25em}\setlength{\parskip}{0pt}}
\setcounter{secnumdepth}{0}
\setlength{\parskip}{0pt}
\let\Tiny=\tiny

% Defnining color scheme options
\usepackage{color, hyperref}

% Primary Colors
\definecolor{UNTGreen}{RGB}{0,123,60}
\definecolor{UNTWhite}{RGB}{255,255,255}
\definecolor{UNTBlack}{RGB}{0,0,0}

% Secondary Colors
\definecolor{UNTGreenS1}{RGB}{0,169,80}
\definecolor{UNTGreenS2}{RGB}{80,158,47}

% Bold Accent Colors
\definecolor{UNTLime}{RGB}{196,214,0}
\definecolor{UNTBlue}{RGB}{0,169,224}
\definecolor{UNTPurple}{RGB}{173,26,172}
\definecolor{UNTCoolGrey}{RGB}{83,86,90}
\definecolor{UNTPink}{RGB}{239,75,129}
\definecolor{UNTDarkGreen}{RGB}{0,103,71}

% Soft Accent Colors
\definecolor{UNTPastelGreen}{RGB}{185,220,210}
\definecolor{UNTPastelPink}{RGB}{255,177,187}

\hypersetup{
	breaklinks=true,
	pdfauthor={A. Jordan Nafa (University of North Texas)},
	pdftitle={PSCI 3300 Political Science Research Methods},
	pdfborder={0 0 0},
	colorlinks=true,
	citecolor=UNTGreen,
	urlcolor=UNTGreen,
	linkcolor=black,
	filecolor=black
}
\urlstyle{same}

% Main Theme Settings
\usetheme[width=1.75cm,hideothersections]{PaloAlto}
\usefonttheme{serif}

\usepackage{fancyvrb}
\newcommand{\VerbBar}{|}
\newcommand{\VERB}{\Verb[commandchars=\\\{\}]}
%\DefineVerbatimEnvironment{Highlighting}{Verbatim}{commandchars=\\\{\}}
% Add ',fontsize=\small' for more characters per line
%\newenvironment{Shaded}{}{}
\DefineVerbatimEnvironment{Highlighting}{Verbatim}{fontsize=\small,commandchars=\\\{\}}
\newenvironment{Shaded}{\linespread{1}}{}
\newcommand{\AlertTok}[1]{\textcolor[rgb]{1.00,0.00,0.00}{#1}}
\newcommand{\AnnotationTok}[1]{\textcolor[rgb]{0.00,0.50,0.00}{#1}}
\newcommand{\AttributeTok}[1]{#1}
\newcommand{\BaseNTok}[1]{#1}
\newcommand{\BuiltInTok}[1]{#1}
\newcommand{\CharTok}[1]{\textcolor[rgb]{0.00,0.50,0.50}{#1}}
\newcommand{\CommentTok}[1]{\textcolor[rgb]{0.00,0.50,0.00}{#1}}
\newcommand{\CommentVarTok}[1]{\textcolor[rgb]{0.00,0.50,0.00}{#1}}
\newcommand{\ConstantTok}[1]{#1}
\newcommand{\ControlFlowTok}[1]{\textcolor[rgb]{0.00,0.00,1.00}{#1}}
\newcommand{\DataTypeTok}[1]{#1}
\newcommand{\DecValTok}[1]{#1}
\newcommand{\DocumentationTok}[1]{\textcolor[rgb]{0.00,0.50,0.00}{#1}}
\newcommand{\ErrorTok}[1]{\textcolor[rgb]{1.00,0.00,0.00}{\textbf{#1}}}
\newcommand{\ExtensionTok}[1]{#1}
\newcommand{\FloatTok}[1]{#1}
\newcommand{\FunctionTok}[1]{#1}
\newcommand{\ImportTok}[1]{#1}
\newcommand{\InformationTok}[1]{\textcolor[rgb]{0.00,0.50,0.00}{#1}}
\newcommand{\KeywordTok}[1]{\textcolor[rgb]{0.00,0.00,1.00}{#1}}
\newcommand{\NormalTok}[1]{#1}
\newcommand{\OperatorTok}[1]{#1}
\newcommand{\OtherTok}[1]{\textcolor[rgb]{1.00,0.25,0.00}{#1}}
\newcommand{\PreprocessorTok}[1]{\textcolor[rgb]{1.00,0.25,0.00}{#1}}
\newcommand{\RegionMarkerTok}[1]{#1}
\newcommand{\SpecialCharTok}[1]{\textcolor[rgb]{0.00,0.50,0.50}{#1}}
\newcommand{\SpecialStringTok}[1]{\textcolor[rgb]{0.00,0.50,0.50}{#1}}
\newcommand{\StringTok}[1]{\textcolor[rgb]{0.00,0.50,0.50}{#1}}
\newcommand{\VariableTok}[1]{#1}
\newcommand{\VerbatimStringTok}[1]{\textcolor[rgb]{0.00,0.50,0.50}{#1}}
\newcommand{\WarningTok}[1]{\textcolor[rgb]{0.00,0.50,0.00}{\textbf{#1}}}

\makeatletter
\beamer@headheight=1.5\baselineskip     %controls the height of the headline, default is 2.5    
\makeatother

\setbeamertemplate{footline}{
	\leavevmode%
	\hbox{%
		\begin{beamercolorbox}[wd=.333333\paperwidth,ht=2.25ex,dp=1ex,center]{author in head/foot}%
			\usebeamerfont{title in head/foot}\insertshortauthor
		\end{beamercolorbox}%
		\begin{beamercolorbox}[wd=.333333\paperwidth,ht=2.25ex,dp=1ex,center]{title in head/foot}%
			\usebeamerfont{title in head/foot}\insertshortinstitute
		\end{beamercolorbox}%
		\begin{beamercolorbox}[wd=.333333\paperwidth,ht=2.25ex,dp=1ex,right]{date in head/foot}%
			\usebeamerfont{date in head/foot}\insertshortdate{}\hspace*{2em}
			\insertframenumber{} / \inserttotalframenumber\hspace*{2ex} 
	\end{beamercolorbox}}%
	\vskip0pt%
}
\makeatother

% Sections and subsections should not get their own damn slide.
\AtBeginPart{}
\AtBeginSection{}
\AtBeginSubsection{}
\AtBeginSubsubsection{}

% Get rid of navigation symbols.
\setbeamertemplate{navigation symbols}{}
\setbeamertemplate{caption}[numbered]

% Some optional color adjustments to Beamer
\setbeamercolor{palette primary}{bg=UNTGreen,fg=white}
\setbeamercolor{palette secondary}{bg=UNTDarkGreen,fg=white}
\setbeamercolor{palette tertiary}{bg=UNTGreenS1,fg=white}
\setbeamercolor{palette quaternary}{bg=UNTGreenS2,fg=white}
\setbeamercolor{frametitle}{bg=UNTGreen,fg=white}

% Sidebar color settings
\setbeamercolor{sidebar}{bg=UNTGreenS1}
\setbeamercolor{author in sidebar}{fg=white}
\setbeamercolor{title in sidebar}{fg=white}
\setbeamercolor{section in sidebar}{fg=white} 
\setbeamercolor{section in sidebar shaded}{fg=white}

\setbeamercolor{structure}{fg=UNTGreen} % itemize, enumerate, etc
%\setbeamercolor{section in toc}{fg=UNTGreen} % TOC sections
%\setbeamercolor{subsection in toc}{fg=UNTDarkGreen} % TOC sections
%\setbeamercolor{subsection in head/foot}{bg=UNTGreenS2,fg=white}
\setbeamercolor{shorttitle in headline}{bg=UNTGreen,fg=white}

\setbeamercolor{section in sidebar}{fg=white}            %color of the active section
\setbeamercolor{section in sidebar shaded}{fg=white}     %color of the inactive section
%\setbeamercolor{subsection in sidebar}{fg=...}         %color of the active subsection
%\setbeamercolor{subsection in sidebar shaded}{fg=...}  %color of the inactive subsection
%\setbeamercolor{title in sidebar}{fg=...}              %color of the presentation title
%
\usepackage{xpatch}
\xpatchcmd{\itemize}{\def\makelabel}{\setlength{\itemsep}{0.25em}\def\makelabel}{}{}

% Adjust some item elements. More cosmetic things.
%-------------------------------------------------

\author{A. Jordan Nafa}
	\title[PSCI 3300 Political Science Research Methods]{Introduction to
Quantitative Political Research}
\institute[University of North Texas]{
	\normalsize{University of North Texas\\ 
		Department of Political Science}
}
\date{August 30th, 2022}
%\subject{}
%\setbeamercovered{transparent}
%\setbeamertemplate{navigation symbols}{}
\begin{document}
	
	% Title Slide
	\begin{frame}
		\maketitle
	\end{frame}

\hypertarget{introduction}{%
\subsection{Introduction}\label{introduction}}

\begin{frame}{Introduction}
\begin{itemize}[<+->]
\item
  \textbf{Quantitative Social Science} is the use of quantitative data
  to study, analyze, or predict social and political phenomena
\item
  Outside of academic settings, this is more commonly known as
  \emph{data science}

  \begin{itemize}[<+->]
  \tightlist
  \item
    The logic of quantitative social inquiry we will cover in this class
    is applied in academic social sciences, government, non-profits, and
    in the private sector
  \end{itemize}
\item
  This course introduces students to the logic of causal inference and
  the tools social scientists use to study social, political, and
  economic phenomena

  \begin{itemize}[<+->]
  \tightlist
  \item
    We will focus primarily on quantitative approaches to the study of
    politics with an emphasis on application, causal reasoning, and
    prediction
  \end{itemize}
\end{itemize}
\end{frame}

\hypertarget{why-take-this-course}{%
\subsection{Why Take this Course?}\label{why-take-this-course}}

\begin{frame}{Why Take this Course?}
\begin{itemize}[<+->]
\item
  PSCI 3300 is a required course for political science majors so you'll
  have to take it at some point
\item
  The skills taught in this course are standard across an increasing
  number of industries and fields research

  \begin{itemize}[<+->]
  \item
    A basic knowledge of statistics, programming, and quantitative
    reasoning prepares you for a range of future career opportunities or
    graduate school
  \item
    Contemporary political science is a primarily quantitative field, so
    understanding the tools of the trade will help you understand and
    critique the things you read in your other classes
  \end{itemize}
\item
  I will not require you to laugh at my jokes
\end{itemize}
\end{frame}

\hypertarget{teaching-philosophy}{%
\subsection{Teaching Philosophy}\label{teaching-philosophy}}

\begin{frame}{Teaching Philosophy}
\begin{itemize}[<+->]
\item
  Math for the sake of math is an unfortunately common but largely
  unproductive way to teach quantitative social science
\item
  A college-level course in fundamentals of computer programming or
  elementary statistics may be helpful but is not required
\item
  Our focus in this course will be primarily on how to apply, interpret,
  and evaluate analyses in political science rather than on the
  statistical theory behind them
\item
  Though there is some basic statistics required, that means less of
  this
  \[\text{Normal}(y|\mu,\sigma) = \frac{1}{\sqrt{2 \pi}\, \sigma} \exp\left( - \, \frac{1}{2}\left(\frac{y -\mu}{\sigma} \right)^2 \right)\]
\end{itemize}
\end{frame}

\begin{frame}[fragile]{Teaching Philosophy}
\protect\hypertarget{teaching-philosophy-1}{}
\begin{itemize}[<+->]
\tightlist
\item
  And more hands-on stuff like this
\end{itemize}

\begin{Shaded}
\begin{Highlighting}[]
\DocumentationTok{\#\# Simulate 10,000 random draws from a standard normal distribution}
\NormalTok{std\_norm }\OtherTok{\textless{}{-}} \FunctionTok{rnorm}\NormalTok{(}\AttributeTok{n =} \FloatTok{10e3}\NormalTok{, }\AttributeTok{mean =} \DecValTok{0}\NormalTok{, }\AttributeTok{sd =} \DecValTok{1}\NormalTok{)}

\DocumentationTok{\#\# Print a summary}
\FunctionTok{summary}\NormalTok{(std\_norm)}
\end{Highlighting}
\end{Shaded}

\begin{verbatim}
       Min.   1st Qu.    Median      Mean   3rd Qu.      Max. 
  -3.514303 -0.679216 -0.035599 -0.009108  0.671100  4.092691
\end{verbatim}
\end{frame}

\hypertarget{course-structure}{%
\subsection{Course Structure}\label{course-structure}}

\begin{frame}{Course Structure}
\begin{itemize}[<+->]
\item
  \textbf{Class Meetings}

  \begin{itemize}[<+->]
  \item
    Tuesday/Thursday from 9:30 AM -- 10:50 AM
  \item
    Part of our course time will be spent on lecture and answering
    questions and the other half will be spent working with data in R
  \item
    You should come to class having done the readings on the syllabus
    for that day so that we can discuss anything that isn't clear or
    appears confusing
  \end{itemize}
\item
  \textbf{Problem Sets}

  \begin{itemize}[<+->]
  \item
    To get hands on experience writing R code, manipulating and modeling
    data, and thinking through the logic of cause and effect, you will
    complete a series of problem sets that ask you to apply the topics
    we cover in class using real data.
  \item
    \emph{You must demonstrate that you made a good faith effort to work
    through each question in order to receive any partial credit for
    incorrect answers}
  \end{itemize}
\end{itemize}
\end{frame}

\hypertarget{required-course-materials}{%
\subsection{Required Course Materials}\label{required-course-materials}}

\begin{frame}{Required Course Materials}
\begin{itemize}[<+->]
\item
  There is one required textbook for this course which can be purchased
  from the UNT campus bookstore or Amazon

  \begin{itemize}[<+->]
  \item
    \fullcite{BuenodeMesquita2021}
  \end{itemize}
\item
  Although it is not required, I also recommend obtaining a copy of

  \begin{itemize}[<+->]
  \item
    \fullcite{Gelman2021}
  \end{itemize}
\item
  All additional readings for the course and various instructional
  resources for applied statistics and programming in R are provided via
  the course's Canvas page
\end{itemize}
\end{frame}

\begin{frame}{Technical Requirements}
\protect\hypertarget{technical-requirements}{}
\begin{itemize}[<+->]
\item
  To complete the requirements for this course you will need access to a
  laptop or desktop computer with a stable internet connection
\item
  If you do not have access to a personal computer that meets the
  minimum requirements necessary to run R, you can access both R and
  RStudio from computers on the UNT campus
\item
  The computers in the political science lab in Wooten Hall 173 should
  have the most recent version of R, RStudio, and RTools
\item
  If you expect to have difficulty meeting the technical requirements
  for this course you need to let me know immediately
\end{itemize}
\end{frame}

\end{document}